\documentclass{article}
\title{Sample Document for iedit Testing}
\author{Test Author}
\date{\today}

\begin{document}

\maketitle

\section{Introduction}

This is a sample document for testing the iedit tool. It contains some gramatical errors and typos that should be corrected. The tool should improve the form, syntax, and grammar while preserving the meaning.

The experiment was conducted with 42 participants, achieving an accuracy of 95.7\%. The p-value was 0.001, which is statistically significant.

\section{Methods}

We used a novel aproach to solve the problem. The method consists of several steps:

\begin{enumerate}
    \item Collect data from 42 sources
    \item Process the data using algorithm A
    \item Analyze the results using statistical methods
    \item Interpret the findings in context
\end{enumerate}

The formula $E = mc^2$ describes the relationship between energy and mass.

\begin{equation}
F = ma
\end{equation}

\section{Results}

The results of our experiment are summarized in Table \ref{tab:results}.

\begin{table}
\centering
\caption{Results of the experiment}
\label{tab:results}
\begin{tabular}{|c|c|c|}
\hline
Method & Accuracy & Time (s) \\
\hline
Method 1 & 92.5\% & 0.45 \\
Method 2 & 87.3\% & 0.32 \\
Method 3 & 95.8\% & 0.67 \\
\hline
\end{tabular}
\end{table}

As we can see from the table, Method 3 achieved the highest accuracy of 95.8\%, but it also took the longest time to run (0.67 seconds).

\section{Conclusion}

In conclusion, our method outperformed existing approaches by a significant margin. Future work will focus on improving the efficiency of the algorithm to reduce the computation time while maintaining the high accuracy.

\end{document}